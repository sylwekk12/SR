% !TeX encoding = UTF-8
% !TeX spellcheck = pl_PL

% $Id:$

%Author: Wojciech Domski
%Szablon do ząłożeń projektowych, raportu i dokumentacji z steorwników robotów
%Wersja v.1.0.0
%


%% Konfiguracja:
\newcommand{\kurs}{Sterowniki robot\'{o}w}
\newcommand{\formakursu}{Projekt}

%odkomentuj właściwy typ projektu, a pozostałe zostaw zakomentowane
\newcommand{\doctype}{Za\l{}o\.{z}enia projektowe} %etap I
%\newcommand{\doctype}{Raport} %etap II
%\newcommand{\doctype}{Dokumentacja} %etap III

%wpisz nazwę projektu
\newcommand{\projectname}{Wykrywacz kradzieży}

%wpisz akronim projektu
\newcommand{\acronim}{WK}

%wpisz Imię i nazwisko oraz numer albumu
\newcommand{\osobaA}{Sylwester \textsc{Kozieja}, 235798}
%w przypadku projektu jednoosobowego usuń zawartość nowej komendy
\newcommand{\osobaB}{Paula \textsc{Langkafel}, 235373}

%wpisz termin w formie, jak poniżej dzień, parzystość, godzina
\newcommand{\termin}{\'sroda TN 13}

%wpisz imię i nazwisko prowadzącego
\newcommand{\prowadzacy}{mgr in\.{z}. Wojciech \textsc{Domski}}

\documentclass[10pt, a4paper]{article}

\include{preambula}
\usepackage{hyperref}	
\begin{document}

\def\tablename{Tabela}	%zmienienie nazwy tabel z Tablica na Tabela

\begin{titlepage}
	\begin{center}
		\textsc{\LARGE \formakursu}\\[1cm]		
		\textsc{\Large \kurs}\\[0.5cm]		
		\rule{\textwidth}{0.08cm}\\[0.4cm]
		{\huge \bfseries \doctype}\\[1cm]
		{\huge \bfseries \projectname}\\[0.5cm]
		{\huge \bfseries \acronim}\\[0.4cm]
		\rule{\textwidth}{0.08cm}\\[1cm]
		
		\begin{flushright} \large
		\emph{Skład grupy:}\\
		\osobaA\\
		\osobaB\\[0.4cm]
		
		\emph{Termin: }\termin\\[0.4cm]

		\emph{Prowadzący:} \\
		\prowadzacy \\
		
		\end{flushright}
		
		\vfill
		
		{\large \today}
	\end{center}	
\end{titlepage}

\newpage
\tableofcontents
\newpage

%Obecne we wszystkich dokumentach
\section{Opis projektu}
\label{sec:OpisProjektu}
%Jak rozumieć moduły
Celem projektu jest stworzenie urządzenia, które poprzez komunikację z akcelerometrem wykrywa niepożądany ruch. Użytkownik będzie miał możliwość wybrania sposobu otrzymywania komunikatów. Jednym z założeń projektu jest wybór opcjonalnego interfejsu audio. Urządzenia ma zawierać menu dające możliwość podstawowej konfiguracji, takiej jak załączenie alarmu i wybór sposobu komunikowania się oraz ustawianie poziomu załączania alarmu. Dodatkową opcją jest wizualizowanie poziomu rejestrowanych przyspieszeń. \\ 
Na zakres prac składa się oprogramowanie zewnętrznej pamięci Flash , skonfigurowanie akcelerometru, a także realizacja komunikacji z interfejsami.

\section{Konfiguracja mikrokontrolera}
\begin{figure}[H]
\centering
\includegraphics[width=13cm]{figures/cub.png}
\caption{Konfiguracja wyjść mikrokontrolera w programie STM32CubeMx}
\end{figure}
\newpage

\begin{figure}[H]
\includegraphics[width=17cm]{figures/zeg.png}
\caption{Konfiguracja zegarów mikrokontrolera}
\end{figure}
\newpage

\subsection{Konfiguracja pinów mikrokontrolera}
\begin{table}[H]
\centering
\begin{tabular}{|l|l|l|l|}
\hline
Numer pinu & Pin & Tryb pracy & Funkcja/ etykieta \\ \hline
1&		PE2&		SAI1\_MCLK\_A	&	SAI1\_MCK	\\
2&		PE3	&	GPIO\_Output		&	AUDIO\_RST\\
3&		PE4	&	SAI1\_FS\_A			&	SAI1\_FS\\
4&		PE5	&	SAI1\_SCK\_A		&	SAI1\_SCK\\
5&		PE6	&	SAI1\_SD\_A	& \\
15&		PC0	&	GPIO\_Output		&	MAG\_CS\\
16&		PC1	&	LCD\_SEG19	& \\
17&		PC2	&	GPIO\_Input	&		MAG\_DRDY\\
18	&	PC3	&	LCD\_VLCD	& \\
23&		PA0	&	GPIO\_EXTI0	&		JOY\_CENTER\\
24	&	PA1	&	GPIO\_EXTI1	&		JOY\_LEFT\\
25&		PA2	&	GPIO\_EXTI2	&		JOY\_RIGHT\\
26	&	PA3	&	GPIO\_EXTI3	&		JOY\_UP\\
30&		PA5	&	GPIO\_EXTI5	&		JOY\_DOWN\\
31	&	PA6	&	LCD\_SEG3	&	\\
32	&	PA7	&	LCD\_SEG4	&	\\
33	&	PC4	&	LCD\_SEG22	&	\\
34	&	PC5	&	LCD\_SEG23	&	\\
35	&	PB0	&	LCD\_SEG5	&	\\
36	&	PB1	&	LCD\_SEG6	&	\\
37	&	PB2	&	GPIO\_Output	&		LED\_R\\
38	&	PE7	&	SAI1\_SD\_B	&		AUODIO\_DIN\\
39	&	PE8	&	GPIO\_Output	&		LED\_G\\
40	&	PE9*&	SAI1\_FS\_B	&		AOUDIO\_CLK\\
41	&	PE10&	QUADSPI\_CLK	&		QSPI\_CLK\\
42	&	PE11&	QUADSPI\_NCS	&		QSPI\_CS\\
43	&	PE12&	QUADSPI\_BK1\_IO0&		QSPI\_D0\\
44	&	PE13&	QUADSPI\_BK1\_IO1	&	QSPI\_D1\\
45	&	PE14&	QUADSPI\_BK1\_IO2	&	QSPI\_D2\\
46	&	PE15&	QUADSPI\_BK1\_IO3	&	QSPI\_D3\\
47	&	PB10&	LCD\_SEG10	&\\
48	&	PB11&	LCD\_SEG11	&\\
51	&	PB12&	LCD\_SEG12	&\\
52	&	PB13&	LCD\_SEG13	&\\
53	&	PB14&	LCD\_SEG14	&\\
54 &	PB15&	LCD\_SEG15	&\\
55	&	PD8	&	LCD\_SEG28	&\\
56	&	PD9	&	LCD\_SEG29	&\\
57	&	PD10&	LCD\_SEG30	& \\
58	&	PD11&	LCD\_SEG31	&\\
59	&	PD12&	LCD\_SEG32	&\\
60	&	PD13&	LCD\_SEG33	&\\
61	&	PD14&	LCD\_SEG34	&\\
62	&	PD15&	LCD\_SEG35	&\\
63	&	PC6	&	LCD\_SEG24	&\\
64	&	PC7	&	LCD\_SEG25	&\\
65	&	PC8	&	LCD\_SEG26	&\\
66	&	PC9	&	LCD\_SEG27	&\\
67	&	PA8	&	LCD\_COM0	&\\
68	&	PA9	&	LCD\_COM1	&\\
69	&	PA10&	LCD\_COM2	&\\
77	&	PA15 &	(JTDI) LCD\_SEG17&	\\
78	&	PC10&	LCD\_SEG40	&\\ \hline

\end{tabular}
\caption{Konfiguracja pinów mikrokontrolera}
\end{table}

\begin{table}[H]
\centering
\begin{tabular}{|l|l|l|l|}
\hline
Numer pinu & Pin & Tryb pracy & Funkcja/ etykieta \\ \hline
79	&	PC11&	LCD\_SEG41	&\\
80	&	PC12&	LCD\_SEG42	&\\
83	&	PD2	&	LCD\_SEG43	&\\
89	&	PB3 &	(JTDO-TRACESWO)	LCD\_SEG7&\\	
90	&	PB4 &	(NJTRST)LCD\_SEG8	&\\
91	&	PB5	&	LCD\_SEG9	&\\
92	&	PB6	&	I2C1\_SCL	&\\
93	&	PB7	&	I2C1\_SDA	&\\
95	&	PB8	&	LCD\_SEG16	&\\
96	&	PB9	&	LCD\_COM3	&\\
97	&	PE0	&	GPIO\_Output	&		XL\_CS\\
98	&	PE1	&	LCD\_SEG37	&\\ \hline

\end{tabular}
\caption{Konfiguracja pinów mikrokontrolera}
\end{table}

\subsection{Konfiguracja peryferiów}
Konfiguracja peryferiów
\begin{enumerate}
\item QUADSPI\\
Interfejs użyty do obsługi pamięci Flash zapewniający dużą przepustowość dzięki czterem liniom danych (PE12:15) oraz dwóm liniom sterującym (PE10 i PE11).
\item LCD\\
Interfejs wyświetlacza umożliwiający komunikację z urządzeniem poprzez zaprojektowane menu. Korzysta z dużej ilości pinów, które opisane są na rysunku z konfiguracją. Ustawienia standardowe,parametr Duty Selection ustawiony na 1/4.
\item GPIO \\
Prosty interfejs do sterowania cyfrowymi wejściami/wyjściami. Użyty do obsługi LED i joystick'a. 5 pinów dla joystick'a (PA0:3 i PA5) oraz pin dla diody LED (PB2).
\item I2C\\
//Coś tu mamy? akcelerometr
\item SAI\\
Interfejs do obsługi wyjścia audio.
\item MEMS\\
albo tu akcelerometr
\end{enumerate}

\section{Urządzenia zewnętrzne}
\subsection{Akcelerometr -- LSM303C}

\section{Założenia projektowe}
\begin{enumerate}
\item Projekt będzie wykonywany w oparciu o płytkę STM32L476 Discovery wypożyczoną od prowadzącego kurs
\item Pomiar przyspieszenia będzie odbywał się przez wbudowany moduł z akcelerometrem
\item W przypadku wykrycia alarmu urządzenie podejmie określone kroki.
\item Menu sterowane z joystick'iem zapewni możliwość wybór ustawień sygnalizacji alarmu(dioda oraz głośnik).
\item Zbieranie danych o alarmach i przechowywanie w pamięci Flash.
\item Wykorzystanie zegara RTC do umiejscowienia zdarzenia alarmu w jego lokalnym czasie.
\item Badania dotyczące wykrywania progu alarmu i sprawdzenie funkcjonalności zaprojektowanego urządzenia.

\end{enumerate}


\section{Harmonogram pracy}

\subsection{Zakres prac}
Zapoznanie się z mikrokontrolerem, konfiguracja peryferiów. Implementacja obsługi zarówno pamięci Flash, jak i akcelerometru. Skonfigurowanie zegara RTC w celu późniejszej implementacji przekazywania godziny nieplanowanego ruchu. Zapoznanie z literaturą i poruszanym problemem. Przeprowadzenie badań na temat progów i optymalizacji działania urządzenia. 
\subsection{Kamienie milowe}
\begin{enumerate}
\item Oddanie I etapu projektu. Projekt powinien zawierać założenia oraz plan co będzie podstawą do rozpoczęcia prac.
\item Oddanie II etapu projektu. Konfiguracja peryferiów powinna być już sfinalizowana, a przynajmniej na etapie pozwalającym rozpoczęcie kolejnego etapu związanego z badaniem przyspieszeń które powinny aktywować alarm.
\item Oddanie III etapu gdzie projekt powinien być już kompletny.
Wg planu projekt powinien zakończyć się tydzień przed ostatecznym terminem złożenia pracy u prowadzącego.
\end{enumerate}
\subsection{Diagram Gantta}


\begin{figure}[H]
	\centering
	\includegraphics[width=1\textwidth]{figures/dg.png}
	\caption{Diagram Gantta}
	\label{fig:DiagramGantta}
\end{figure}

%Obecne w dokumencie do etapu I

\subsection{Podział pracy}
Oboje uczestnicy projektu zajmą się wstępną konfiguracją peryferiów odbywającą się za pomocą programu CubeMx. Zostanie zaimplementowana obsługa pamięci Flash, a także akcelerometru. W tym czasie opracowany zostanie również sposób przechowywania danych na zewnętrznej pamięci Flash. Projekt menu ma zakładać możliwość wyboru sygnału uruchamiającego alarm.

\begin{table}[H]
	\centering
	\begin{tabular}{|L{7cm}|L{0.8cm}||L{7cm}|L{0.8cm}|}
		\hline
		\hline
		\textbf{Sylwester Kozieja} & 
		\% & 
		\textbf{Paula Langkafel} & \%\\
		\hline
		\hline
		Wstępna konfiguracja peryferiów w programie CubeMx		& &	
		Wstępna konfiguracja peryferiów w programie CubeMx &\\
		\hline
		Wstępny projekt menu& &
		Implementacja obsługi akcelerometru &\\
		
		\hline Konfiguracja zegara RTC & & &\\
		\hline
	\end{tabular}
	\caption{Podział pracy -- Etap II}
	\label{tab:PodzialPracyEtap2}
\end{table}

\begin{table}[H]
	\centering
	\begin{tabular}{|L{7cm}|L{0.8cm}||L{7cm}|L{0.8cm}|}
		\hline
		\hline
		\textbf{Sylwester Kozieja} & 
		\% & 
		\textbf{Paula Langkafel} & \%\\
		\hline
		\hline
		Oprogramowanie zewnętrznej pamięci Flash oraz opracowanie sposobu przechowywania danych		& &	
		Opracowanie kryteriów wykrywania alarmu &\\
		\hline
		Implementacja opracowanych rozwiązań wykrywania alarmu  & &
		Wykonanie testów urządzenia &\\
		\hline
		Sygnalizacja audiowizualna za pomocą Audio DAC oraz diody LED & & &\\
		\hline
	\end{tabular}
	\caption{Podział pracy -- Etap III}
	\label{tab:PodzialPracyEtap3}
\end{table}


\newpage
%\section{Literatura}
%\begin{enumerate}
%\item Vehicle Tracker wih a GPS and Accelerometer Sensor System in Jakarta - Suryadiputra Liawatimena, Jimmy Linggarjati
%\item UM1928 User manual - Getting started with STM32L476G discovery kit software development tools \\ \url{https://www.st.com/content/ccc/resource/technical/document/user_manual/c8/6c/20/e4/8c/3c/4d/13/DM00217936.pdf/files/DM00217936.pdf/jcr:content/translations/en.DM00217936.pdf}
%\item Quantitative Accelerated Life Testing of MEMS Accelerometers - Marius Bazu, Lucian Galateanu, Virgil Emil Ilian, Jerome Loicq, Serge Habraken,
%Jean-Paul Collette
%\item Overview of ST-LINK derivatives \\ \url{https://www.st.com/content/ccc/resource/technical/document/technical_note/group0/30/c8/1d/0f/15/62/46/ef/DM00290229/files/DM00290229.pdf/jcr:content/translations/en.DM00290229.pdf}
%\item 

%\end{enumerate}

\nocite{*}
\addcontentsline{toc}{section}{Bibilografia}
\bibliography{bibliografia}
\bibliographystyle{plabbrv}



\end{document}



