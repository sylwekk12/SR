% !TeX encoding = UTF-8
% !TeX spellcheck = pl_PL

% $Id:$

%Author: Wojciech Domski
%Szablon do ząłożeń projektowych, raportu i dokumentacji z steorwników robotów
%Wersja v.1.0.0
%


%% Konfiguracja:
\newcommand{\kurs}{Sterowniki robot\'{o}w}
\newcommand{\formakursu}{Projekt}

%odkomentuj właściwy typ projektu, a pozostałe zostaw zakomentowane
\newcommand{\doctype}{Za\l{}o\.{z}enia projektowe} %etap I
%\newcommand{\doctype}{Raport} %etap II
%\newcommand{\doctype}{Dokumentacja} %etap III

%wpisz nazwę projektu
\newcommand{\projectname}{Wykrywacz kradzieży}

%wpisz akronim projektu
\newcommand{\acronim}{WK}

%wpisz Imię i nazwisko oraz numer albumu
\newcommand{\osobaA}{Sylwester \textsc{Kozieja}, 235798}
%w przypadku projektu jednoosobowego usuń zawartość nowej komendy
\newcommand{\osobaB}{Paula \textsc{Langkafel}, 235373}

%wpisz termin w formie, jak poniżej dzień, parzystość, godzina
\newcommand{\termin}{\'sroda TN 13}

%wpisz imię i nazwisko prowadzącego
\newcommand{\prowadzacy}{mgr in\.{z}. Wojciech \textsc{Domski}}

\documentclass[10pt, a4paper]{article}

\include{preambula}
\usepackage{hyperref}	
\begin{document}

\def\tablename{Tabela}	%zmienienie nazwy tabel z Tablica na Tabela

\begin{titlepage}
	\begin{center}
		\textsc{\LARGE \formakursu}\\[1cm]		
		\textsc{\Large \kurs}\\[0.5cm]		
		\rule{\textwidth}{0.08cm}\\[0.4cm]
		{\huge \bfseries \doctype}\\[1cm]
		{\huge \bfseries \projectname}\\[0.5cm]
		{\huge \bfseries \acronim}\\[0.4cm]
		\rule{\textwidth}{0.08cm}\\[1cm]
		
		\begin{flushright} \large
		\emph{Skład grupy:}\\
		\osobaA\\
		\osobaB\\[0.4cm]
		
		\emph{Termin: }\termin\\[0.4cm]

		\emph{Prowadzący:} \\
		\prowadzacy \\
		
		\end{flushright}
		
		\vfill
		
		{\large \today}
	\end{center}	
\end{titlepage}

\newpage
\tableofcontents
\newpage

%Obecne we wszystkich dokumentach
\section{Opis projektu}
\label{sec:OpisProjektu}
%Jak rozumieć moduły
Celem projektu jest stworzenie urządzenia, które poprzez komunikację z akcelerometrem wykrywa niepożądany ruch. Użytkownik będzie miał możliwość wybrania sposobu otrzymywania komunikatów. Jednym z założeń projektu jest wybór opcjonalnego interfejsu audio. Urządzenia ma zawierać menu dające możliwość podstawowej konfiguracji, takiej jak załączenie alarmu i wybór sposobu komunikowania się oraz ustawianie poziomu załączania alarmu. Dodatkową opcją jest wizualizowanie poziomu rejestrowanych przyspieszeń. \\ 
Na zakres prac składa się oprogramowanie zewnętrznej pamięci Flash , skonfigurowanie akcelerometru, a także realizacja komunikacji z interfejsami.


\section{Założenia projektowe}
\begin{enumerate}
\item Projekt będzie wykonywany w oparciu o płytkę STM32L476 Discovery wypożyczoną od prowadzącego kurs
\item Pomiar przyspieszenia będzie odbywał się przez wbudowany moduł z akcelerometrem
\item W przypadku wykrycia alarmu urządzenie podejmie określone kroki.
\item Menu sterowane z joystick'iem zapewni możliwość wybór ustawień sygnalizacji alarmu(dioda oraz głośnik).
\item Zbieranie danych o alarmach i przechowywanie w pamięci Flash.
\item Wykorzystanie zegara RTC do umiejscowienia zdarzenia alarmu w jego lokalnym czasie.
\item Badania dotyczące wykrywania progu alarmu i sprawdzenie funkcjonalności zaprojektowanego urządzenia.

\end{enumerate}


\section{Harmonogram pracy}

\subsection{Zakres prac}
Zapoznanie się z mikrokontrolerem, konfiguracja peryferiów. Implementacja obsługi zarówno pamięci Flash, jak i akcelerometru. Skonfigurowanie zegara RTC w celu późniejszej implementacji przekazywania godziny nieplanowanego ruchu. Zapoznanie z literaturą i poruszanym problemem. Przeprowadzenie badań na temat progów i optymalizacji działania urządzenia. 
\subsection{Kamienie milowe}
\begin{enumerate}
\item Oddanie I etapu projektu. Projekt powinien zawierać założenia oraz plan co będzie podstawą do rozpoczęcia prac.
\item Oddanie II etapu projektu. Konfiguracja peryferiów powinna być już sfinalizowana, a przynajmniej na etapie pozwalającym rozpoczęcie kolejnego etapu związanego z badaniem przyspieszeń które powinny aktywować alarm.
\item Oddanie III etapu gdzie projekt powinien być już kompletny.
Wg planu projekt powinien zakończyć się tydzień przed ostatecznym terminem złożenia pracy u prowadzącego.
\end{enumerate}
\subsection{Diagram Gantta}


\begin{figure}[H]
	\centering
	\includegraphics[width=1\textwidth]{figures/dg.png}
	\caption{Diagram Gantta}
	\label{fig:DiagramGantta}
\end{figure}

%Obecne w dokumencie do etapu I

\subsection{Podział pracy}
Oboje uczestnicy projektu zajmą się wstępną konfiguracją peryferiów odbywającą się za pomocą programu CubeMx. Zostanie zaimplementowana obsługa pamięci Flash, a także akcelerometru. W tym czasie opracowany zostanie również sposób przechowywania danych na zewnętrznej pamięci Flash. Projekt menu ma zakładać możliwość wyboru sygnału uruchamiającego alarm.

\begin{table}[H]
	\centering
	\begin{tabular}{|L{7cm}|L{0.8cm}||L{7cm}|L{0.8cm}|}
		\hline
		\hline
		\textbf{Sylwester Kozieja} & 
		\% & 
		\textbf{Paula Langkafel} & \%\\
		\hline
		\hline
		Wstępna konfiguracja peryferiów w programie CubeMx		& &	
		Wstępna konfiguracja peryferiów w programie CubeMx &\\
		\hline
		Implementacja obsługi pamięci Flash & &
		Implementacja obsługi akcelerometru &\\
		\hline
		Opracowanie sposobu przechowywania danych na zewnętrznej pamięci FLASH & &
		Wstępny projekt menu  & \\
		\hline
		Sygnalizacja audiowizualna za pomocą Audio DAC oraz diody LED & & Konfiguracja zegara RTC &\\
		\hline
	\end{tabular}
	\caption{Podział pracy -- Etap II}
	\label{tab:PodzialPracyEtap2}
\end{table}

\begin{table}[H]
	\centering
	\begin{tabular}{|L{7cm}|L{0.8cm}||L{7cm}|L{0.8cm}|}
		\hline
		\hline
		\textbf{Sylwester Kozieja} & 
		\% & 
		\textbf{Paula Langkafel} & \%\\
		\hline
		\hline
		Oprogramowanie zewnętrznej pamięci Flash		& &	
		Opracowanie kryteriów wykrywania alarmu &\\
		\hline
		Implementacja opracowanych rozwiązań wykrywania alarmu  & &
		Wykonanie testów urządzenia &\\
		\hline
	\end{tabular}
	\caption{Podział pracy -- Etap III}
	\label{tab:PodzialPracyEtap3}
\end{table}


\newpage
%\section{Literatura}
%\begin{enumerate}
%\item Vehicle Tracker wih a GPS and Accelerometer Sensor System in Jakarta - Suryadiputra Liawatimena, Jimmy Linggarjati
%\item UM1928 User manual - Getting started with STM32L476G discovery kit software development tools \\ \url{https://www.st.com/content/ccc/resource/technical/document/user_manual/c8/6c/20/e4/8c/3c/4d/13/DM00217936.pdf/files/DM00217936.pdf/jcr:content/translations/en.DM00217936.pdf}
%\item Quantitative Accelerated Life Testing of MEMS Accelerometers - Marius Bazu, Lucian Galateanu, Virgil Emil Ilian, Jerome Loicq, Serge Habraken,
%Jean-Paul Collette
%\item Overview of ST-LINK derivatives \\ \url{https://www.st.com/content/ccc/resource/technical/document/technical_note/group0/30/c8/1d/0f/15/62/46/ef/DM00290229/files/DM00290229.pdf/jcr:content/translations/en.DM00290229.pdf}
%\item 

%\end{enumerate}

\nocite{*}
\addcontentsline{toc}{section}{Bibilografia}
\bibliography{bibliografia}
\bibliographystyle{plabbrv}



\end{document}



